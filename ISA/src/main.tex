% LTeX: language=es

\documentclass{article}
\usepackage{geometry}
\usepackage{graphicx}
\usepackage[spanish]{babel}
\usepackage[T1]{fontenc}
\usepackage{csquotes}
\usepackage[style=apa, backend=biber]{biblatex}

\addbibresource{./src/main.bib}

\graphicspath{ {./res/} }

\begin{document}

% LTeX: language=es
% vi:ft=tex

\begin{titlepage}
    \begin{center}
        \Large
        \textbf{PONTIFICIA UNIVERSIDAD CATÓLICA DEL PERÚ}

        \textbf{FACULTAD DE CIENCIAS E INGENIERÍA}

        \textbf{INGENIERÍA INFORMÁTICA}

        \vspace{0.5cm}
        \includegraphics[scale=0.8]{pucp}

        \vspace{1.5cm}
        \textbf{ISA - Informe de Selección de Algoritmos}
    \end{center}
    \large
    \vspace{1cm}

    \textbf{Profesor del curso: } Abraham Eliseo Dávila Ramón
    \vspace{0.5cm}

    \textbf{Equipo: } 5E
    \vspace{0.5cm}

    \textbf{Horario: } 0981
    \vspace{0.5cm}

    \textbf{Integrantes:}
    \vspace{0.5cm}

    \begin{itemize}
        \item{Iván Sebastián Córdova Rivero - 20181923- sebastian.cordova@pucp.edu.pe}
        \item{Oscar Daniel Navarro Cieza - 20186008 - oscar.navarro@pucp.edu.pe}
        \item{Carlos Santos Toro Vera - 20171878 - toro.carlos@pucp.edu.pe}
        \item{Jherson Jair Zuñiga Salas - 20145795 - jherson.zuniga@pucp.edu.pe}
    \end{itemize}

    \begin{center}
        \vspace*{2cm}
        \textbf{Lima, 11 de abril del 2023}
    \end{center}
\end{titlepage}



\tableofcontents

\cleardoublepage{}

\section{Introducción}
En este documento se describirán algoritmos de búsqueda del camino más corto entre
dos puntos en un mapa o grafo. Adicionalmente, se contrastarán los beneficios y
limitaciones de cada algoritmo y la factibilidad de la implementación para este
proyecto del curso Desarrollo de Programas I. Finalmente, un algoritmo será
seleccionado a partir de los criterios previamente desarrollados.

\section{Algoritmos}
\subsection{Dijkstra}
El algoritmo de Dijkstra fue creado por E.W. Dijkstra. La idea principal de este
algoritmo es adoptar la estrategia algorítmica codiciosa para la mejor opción y
viajar al vértice adyacente con el menor peso acumulado y que no haya sido visitado
anteriormente iniciando del vértice inicial \autocite{tongpan}.

\subsection{Best First Search}
La idea principal del algoritmo BFS es utilizar una función heurística para elegir
el siguiente vértice a recorrer. Esta función heurística se define como la distancia
del vértice hasta el vértice objetivo \autocite{mehta}.

\subsection{A*}
El algoritmo A* combina dos criterios para la selección del siguiente vértice. Este
algoritmo mide el peso de cada vértice respecto al origen, también toma en cuenta
una función heurística para medir la distancia entre un vértice y el vértice objetivo.

\section{Comparacion}
El algoritmo de Dijkstra encuentra el camino más corto desde un punto inicial hasta
un objetivo sin falla \autocite{minhang}. El problema de este algoritmo es que siempre
va a tener que tomar en cuenta todos los vértices vecinos durante el recorrido
hasta llegar al destino.

El algoritmo BFS es más rápido que el algoritmo de Dijkstra, ya que ahorra recursos
utilizando la función heurística. Sin embargo, el camino seleccionado por este
algoritmo no necesariamente va a ser el más corto si es que la función heurística
no fue definida correctamente.

El algoritmo A* ha sido método más probado y utilizado como base para problemas
de búsqueda de caminos \autocite{foead}. El algoritmo combina la habilidad del
algoritmo de Dijkstra de siempre encontrar el camino más corto, y utiliza una
función heurística como el algoritmo BFS. El algoritmo A* combina todos los
beneficios del algoritmo Dijkstra y BFS sin padecer de los inconvenientes
de estos \autocite{mehta}.

\section{Conclusiones}
Cuando la topología de rutas es relativamente pequeña, los algoritmos A*, Dijkstra
y BFS el tiempo de ejecución es casi igual \autocite{minhang}. Cuando comienzan
a aumentar la cantidad de nodos que se deben de viajar entonces el algoritmo A*
tiene el tiempo de ejecución mayor.

Particularmente, en el proyecto que se va a desarrollar en el curso de Desarrollo
de Programas I, la cantidad de nodos que se van a tener que atravesar será alto
por lo que se decidió utilizar el algoritmo A* para el desarrollo de este proyecto.

\printbibliography{}

\end{document}
